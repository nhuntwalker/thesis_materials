\section{Introduction}

The structure of the \mw\ holds clues to the processes of formation and evolution of galaxies.
Historical models typically assumed three discrete components described by simple analytic 
expressions: the thin disk, thick disk, and halo \citep{1980ApJS...44...73B, 1989ARA&A..27..555G,1993ARA&A..31..575M}. 
Recent surveys, such as the Sloan Digital Sky Survey \citep[\sdss, ][]{2000AJ....120.1579Y} 
and the Two Micron All Sky Survey \citep[\twomass, ][]{2006AJ....131.1163S}, have provided 
much more detail about these components. For example, SDSS data have constrained stellar distributions 
in the 7-dimensional space spanned by spatial coordinates \citep{2008ApJ...673..864J}, velocity components \citep{2010ApJ...716....1B}, and metallicity \citep{2008ApJ...684..287I}. The resulting maps revealed rich, complex substructure in the distribution of the \mw's stars \citep[e.g.][]{2000AJ....120..963I,2000ApJ...540..825Y,2001ApJ...554L..33V,2002ApJ...569..245N,2003ApJ...599.1082M,2006ApJ...642L.137B,2006ApJ...651L..29G,2006AJ....132..714V}, deeply shaking the older view of a smooth Galaxy. 


The large distance limit of about 100 kpc for some halo populations (e.g. RR Lyrae and BHB stars 
with SDSS, \citealt{2010ApJ...708..717S} and red giants with 2MASS, \citealt{2003ApJ...599.1082M}), as well as about 10 kpc 
for very numerous main sequence stars, is not reached when studying disk component at low galactic
latitudes. As shown, for example, by \cite{2012ApJ...757..166B}, the extinction due to interstellar dust limits 
the SDSS sample of main sequence stars to heliocentric distances of only a few kpc. In order
to avoid detrimental effect of dust extinction, stellar samples need to be probed at longer infrared
wavelengths. Several infrared surveys covering the Galactic plane have recently become available (WISE, \citealt{2010AJ....140.1868W, 2012wise.rept....1C}, GLIMPSE \citealt{2009PASP..121..213C, 2003PASP..115..953B}, VVV \citealt{2012A&A...537A.107S}).

Among populations suitable for studying Galactic structure using infrared surveys, Asymptotic Giant 
Branch (\agb) stars stand out. \agb\ stars represent the last stage of evolution for stars between 0.8 and 8 $M_\odot$ \citep{1983ARA&A..21..271I, 2005ARA&A..43..435H}. Stars from this mass range can reach the final stages of stellar evolution within the Galactic timescale ($\sim$10 Gyr, \citealt{1983ARA&A..21..271I}) and thus are bound to reside throughout the Galaxy wherever other stars are present. This phase of stellar evolution is marked by two distinct
episodes with different observational characteristics: the early \agb\, phase (E-\agb)  and the thermally-pulsing \agb\, phase (TP). During the thermally-pulsing phase, \agb\, stars produce substantial dust-driven stellar winds \citep[$10^{-7} < \dot{M} < 10^{-4}$ $M_\odot$ yr$^{-1}$,][]{2002A&A...391.1053O} rich in oxides (SiO, Al$_2$O$_3$, etc.) or carbon-rich molecules (SiC, AmC, etc.) The dominant chemical composition is highly dependent upon the metallicity of the host galaxy \citep{2005A&A...434..691M}. High-metallicity galaxies like  the Milky Way have a substantial population of oxygen-rich AGB stars \citep{1985A&A...152L...1H}, whereas low-metallicity galaxies such as the Magellanic clouds are dominated by carbon-rich \agb\ stars \citep{2011AJ....142..103B}. In both cases, the other species of AGB star is rarely seen, as richness in one chemical type (e.g. oxides)  necessitates the almost complete capture of the other chemical type (e.g. carbon) in CO \citep{1983ARA&A..21..271I}. 

The dust-rich winds in TP phase  create vast circumstellar shells that are warmed by the stellar photosphere and emit
copiously in the near- and mid-infrared (NIR \& MIR, respectively). Together with high bolometric luminosity
(10$^3$--10$^4$ \Mo), this redistribution of the output radiation to the infrared wavelength makes \agb\ stars
excellent disk probe when infrared survey data are available. Indeed, they were detected all the way to the 
Galactic center even with the IRAS survey \citep{2002MNRAS.337..749J}. Such disk studies can now be significantly
improved thanks to the much more sensitive WISE survey. 

The \emph{Wide-field Infrared Survey Explorer}, \wise, is a space observatory that has imaged essentially the entire sky in the MIR (3.4, 4.6, 12, and 22 $\mu$m). The \wise\, catalog has been positionally matched to the \twomass\ catalog, with the matched catalog listing NIR and MIR 7-band photometry for
hundreds of millions of sources. Given the depths of the two surveys, this catalog should contain AGB stars to many
kpc beyond the Galactic center. Here we develop selection methods for \agb\ stars using WISE and 2MASS data,
and analyse the resulting samples. 

In Section~\ref{sec:data}, we describe in detail the WISE, 2MASS and other auxiliary data used in our study and the data reduction process. 
In Section~\ref{sec:criteria}, we use samples of known Galactic and Magellanic \agb\, stars to derive 
WISE-2MASS color-based selection criteria, and calibrate color-absolute magnitude relations. 
In Section~\ref{sec:distribution}, we describe the spatial density distribution of selected \agb\, candidates
from the Milky Way. Our conclusions are summarized in Section~\ref{sec:conclusions}.
