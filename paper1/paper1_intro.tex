\section{Introduction}


- brief intro to Gal structure

- SDSS and other recent results do not go into the plane

- mention a few studies that do (GLIMPSE etc) and ongoing data taking with DECam

- AGB stars: luminous in IR, relatively easy to find; mention MACHO and OGLE studies, 
   refer to Jackson et al. IRAS-based study, lead to WISE

- brief description of WISE

- paper outline




The formation of galaxies like the \mw\, was long thought to be a steady process that created a smooth distribution of stars. This standard view was exemplified by the \cite{1980ApJS...44...73B} and \cite{1989ARA&A..27..555G} models, described in detail by, e.g., \cite{1993ARA&A..31..575M}. It was further motivated by observations of other galaxies, as well as what little information was available from \emph{High Precision Parallax Collecting Satellite} \citep[\emph{HIPPARCOS}, ][]{1984SSRv...39....1K} and smaller pencil-beam surveys. In these, the \mw\, is modeled by three discrete components described by simple analytic expressions: the thin disk, thick disk, and halo. 

Review is \citet{IBJ2012} \citep{IBJ2012}


The advent of the \emph{Sloan Digital Sky Survey} \citep[\sdss, ][]{2000AJ....120.1579Y} alleviated these limitations, providing accurate digital multi-band optical photometry across a quarter of the sky, as well as the largest optical spectroscopic catalog thus far known. This new influx of data enabled the development and application of photometric parallax methods, using color-magnitude relations to estimate stellar distances. In turn, this led to the large scale ``tomography" of the \mw\; via stellar distributions in the 7-dimensional space spanned by spatial coordinates \citep{2008ApJ...673..864J}, velocity components \citep{2010ApJ...716....1B}, and metallicity \citep{2008ApJ...684..287I}. The resulting maps revealed rich, complex substructure in the distribution of the \mw's stars \citep[e.g.][]{2000AJ....120..963I,2000ApJ...540..825Y,2001ApJ...554L..33V,2002ApJ...569..245N,2003ApJ...599.1082M,2006ApJ...642L.137B,2006ApJ...651L..29G,2006AJ....132..714V}, deeply shaking the existing view of the Galaxy. 



In order to move forward from where \sdss\, tomography left off, we require observations that span an area larger than that of \sdss\, with Galactic objects that can be seen through interstellar dust out to large distances. Stars from the Asymptotic Giant Branch (\agb) are perfect candidates as probes to the \mw. \agb\, stars are represent the last stage of evolution for stars between 0.8 and 8 $M_\odot$ \citep{1983ARA&A..21..271I, 2005ARA&A..43..435H}, so they are bound to reside throughout the galaxy wherever other stars are present. During this phase, they produce substantial stellar winds \citep[$10^{-7} < \dot{M} < 10^{-4} M_\odot$ yr$^{-1}$,][]{2002A&A...391.1053O} rich in SiO and amorphous carbon as they progress through being oxygen-rich to being carbon-rich. These winds collect into circumstellar shells that, when warmed by the stellar photosphere, shine brightly in the near- and mid-infrared (NIR \& MIR respectively). 

The \emph{Wide-field Infrared Survey Explorer} \citep[\wise, ][]{2010AJ....140.1868W, 2012wise.rept....1C} is a space-based observatory that has imaged the entire sky in the MIR (3.4, 4.6, 12, and 22$\mu$m). Additionally, \wise\, has been positionally matched to the Two-Micron All-Sky Survey (\twomass), a four-year mission characterizing the full sky in the NIR. Thus, the \wise\, catalog presents with hundreds of millions of sources with photometry of unprecedented sensitivity in the NIR and MIR--ideal for capturing \agb\, stars at Galactic distances. In Section~\ref{sec:data}, we describe in detail the data that we use for our study. 

In Section~\ref{sec:criteria}, we describe the color-color criteria used to isolate \agb\, stars in the \wise\, dataset, and the color-magnitude relationships that were derived for these stars from the Large Magellanic Cloud and the \mw\, bulge.
In Section~\ref{sec:distribution}, we describe the spatial density distribution of \agb\, candidates in the Milky Way.
Our conclusions can be found in section~\ref{sec:conclusions}.
