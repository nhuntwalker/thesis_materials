\begin{abstract}
We study the structure of the Milky Way disk with candidate asymptotic giant branch (AGB) stars selected from the \emph{Wide-field Infrared Survey Explorer} (\emph{WISE}) catalog. The advantages of our approach, compared to most recent similar works such as those based on SDSS data, are large distance limits due to the high luminosity of AGB stars, small interstellar dust obscuration due to longer wavelengths, and the all-sky coverage of the \emph{WISE} survey. The candidate AGB stars are color-selected with high completeness and low contamination, as quantified using samples of known AGB stars and other objects with known classifications from the SIMBAD and \emph{Sloan Digital Sky Survey} (\emph{SDSS}) databases. Distances to candidate AGB stars are estimated simultaneously with interstellar dust extinction along the line of sight using a 3-dimensional dust distribution model developed to support simulations for the \emph{Large Synoptic Survey Telescope} (\emph{LSST}) and a color -- absolute magnitude relation calibrated using the Large Magellanic Cloud (LMC) and the Galactic bulge. We find that the Galactic disk extends radially out to 15 kpc, with flaring of the disk towards its edge. We present measurements of the vertical scale height and horizontal scale length for double-exponential disk models. We find that the density distribution of AGB candidates within 9 kpc from the Galactic center is consistent with that of a double-exponential profile, while at larger radii the distribution is indistinguishable from a single-exponential profile.
\end{abstract}
