\title{AGB Stars in the IR: Physical Characteristics, Galactic Spatial Distribution, and Chemical Impact of the Milky Way AGB Population}
\author{Nicholas Hunt-Walker}
\maketitle

\section{Physical/Observational Characteristics}
What do I want to talk about here? Absolutely must learn and talk about {\tt DUSTY} here, as well as in section 3.  Estimating mass, temperature, Mbol, mass loss rate, main pulsation period, amount/extent/temperature of circumstellar shell, age, chemical class if possible.  Want some color-color and color-mag diagrams here too.  Probably also talk about candidate selection.

Look into the OGLE and MACHO samples to see if there's some relation between pulsation period and WISE color.  Probably is.

Figure out some way to tell chemical type of AGB star for rest of un-typed candidate sample, which is MOST of the sample.

\section{Spatial Distribution}
I want to talk about my maps here.  Want XYZ 2-axis maps, but also want a 3-D map, as well as a 2D gal l, gal b map.  Also talk about using {\tt TRILEGAL} and using GLIMPSE. Somehow compare my actual distribution of AGB stars to models of galactic structure.  Note of course that my extinction calculations are based entirely on LSST code, so there would be bias.

Consider creating comparable XYZ maps for untyped AGB candidates using both the O-rich and C-rich color-mag relations.

\section{Chemical Impact}
Something about ISM enrichment and chemical output over time. Link the mass loss rate to the chemical content of the circumstellar shell.  Perhaps calculate the chemical enrichment as a function of distance from the star?

\section{Time Frame}
2 years from the end of the general exam.
Major points to hit:
\begin{itemize}
\item Selection of AGB candidates from WISE data
\item Split into high likelihood O-rich and C-rich populations.
\item Figure out someway to relax selection criteria to potentially get more stars.  Consider looking at candidates from both WISE All-Sky and ALLWISE (confusing).
\end{itemize}